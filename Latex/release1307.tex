\documentclass{swp}
\usepackage{hyperref}
\usepackage{amsmath}
\usepackage{amssymb}

\begin{document}

\maketitle{Release 13.07.2015}{13.07.2015}{Paul Eisenhuth}
\\\\\\\\\\

\section{Inhalt:}
\begin{itemize} 
\item Drupal Modul \glqq aae\_{}data\grqq{} Version 1.3
\item Drupal Theme \glqq aae\_{}theme\grqq{} Version 1.1
\end{itemize}
\section{\"Anderungen:}
\begin{itemize} 
\item Attribut \glqq gps\grqq{} geh\"ort jetzt zur Adresstabelle
\item Akteure und Events werden aus zwei .json-Dateien bei der Modulinstallation eingelesen
\item Einzelprofile von Akteuren werden angezeigt
\item Editier- und Erstellrechte werden ber\"ucksichtigt bei Akteuren
\item Events werden aufgelistet, au{\ss}erdem existiert eine Detailansicht\\
\item Im Theme wurde ein Slider anstatt der Karte eingef\"ugt
\item Eine dynamische Akteursansicht ersetzt die generische Eventplatzhalter
\end{itemize}
\section{Installationsanleitung:}
\subsection{Vorraussetzung:}
Eine lauffa\"ahige Version von Drupal 7, empfohlen wird Drupal 7.37. Eine Downloadbeschreibung der Installation findet sich unter \url{https://www.drupal.org/start}.
\subsection{Installation:}
siehe extra Dokument!
\section{Funktionen:}
Bei Installation wird die ben\"otigte Datenstruktur implementiert, bei einer Deinstallation mit allen Daten wieder gel\"oscht. In der Navigationsleiste tauchen Links auf, um Akteure und Events anzuzeigen. \"Uber diese Listen kann man auch die Detailansichten erreichen. \"Uber die Detailansicht eines Akteurs kann dieser auch editiert werden, entsprechende Schreibrechte vorausgesetzt.
\section{Bekannte Probleme:}
Auf der Editierseite werden die bisherigen Eintr\"age nicht zuverl\"assig angezeigt.

\end{document}
