\documentclass{swp}
\usepackage{hyperref}
\usepackage{amsmath}
\usepackage{amssymb}

\begin{document}

\maketitle{Arbeitsstand 15.09.2015}{15.09.2015}{Ruth von Borell}
\\\\\\\\\\

\begin{itemize} 
\item Drupal 7 wurde installiert
\item weiterhin implementiert:
\begin{itemize} 
\item Modul \glqq aae\_{}data\grqq{}: Modul legt Tabellen in der Datenbank an und stellt ein Formular zur Dateneingabe von Akteuren und Events bereit, sowie Seiten zur Anzeige aller Akteure und Events. Desweiteren stellt es Bearbeitungs- und L\"oschenfunktionen dieser zur Verf\"ugung. Au{\ss}erdem gibt es die M\"oglichkeit, Events in einem Kalender anzuzeigen und diese im .ics-Format zu exportieren.
\item Theme \glqq aae\_{}theme\grqq{}: Realisierung eines Designs mit einem Drupaltheme in Absprache mit leipziger-ecken.de.
\end{itemize}
\item Nutzerrollen wurden in Drupal angelegt (Akteur, registrierter User) und entsprechende Berechtigungen gesetzt
\item Karte mittels Mapbox aus Open Street Map wird vorbereitet und beinhaltet bereits die Stadtteilgrenzen und Beispielkoordinaten, siehe \url{http://grinch.pavo.uberspace.de/lo/}
\item Entwicklung eines Datenmodells\\\\
\end{itemize}
\end{document}
